\section{Table of contents}
% 2
\begin{frame}
    \begin{itemize}
        \item<1> Background
        \item<1> Cryptography Security goals
            \begin{itemize}
                \item<1> CIA
            \end{itemize}
        \item<1> Introduction of PKI
            \begin{itemize}
                \item<1> Private Keys
                \item<1> Public Keys
                \item<1> Signatures and Certificates
                \item<1> Transport Layer Security
                \item<1> Diffie-Hellman-Key-Exchange
                \item<1> Certificate Authority
            \end{itemize}
        \item<1> Introduction of WoT
            \begin{itemize}
                \item<1> Direct Trust
                \item<1> Trust Tree
                \item<1> Trust Network
                \item<1> PGP
            \end{itemize}
        \item<1> Trust in Social Networks
        \item<1> Experimental Results and Analyses
        \item<1> Conclusion
    \end{itemize}
\end{frame}

% 3
\section{Background}
\begin{frame}
    \centering
    \includegraphics<1>[width=.8\textwidth, page=1]{pics/media.png}
    \footnote{[Online; accessed 15-Jan-2024] https://www.linkedin.com/pulse/decentralised-everything-vidar-andersen/}
    \begin{itemize}
        \item The diagram above illustrates the lifecycle of a centralized platform.
        \item As the centralized platform develops in an S-curve, the power of the platform over the participants rises, gradually weakening the users' control over their personal information.
        \item Once it reaches the top of the S-curve, the relationship between the centralized platform and its users shifts from Positive-sum to Zero-sum.
    \end{itemize}
\end{frame}

\section{Cryptography Security goals}
% 4
\begin{frame}
    \centering
    \includegraphics<1>[width=.5\textwidth, page=1]{pics/cia.png}
    \footnote{[Online; accessed 15-Jan-2024] https://www.xplore-dna.net/mod/page/view.php?id=6172}
    
    \begin{itemize} 
        \item The basic Cryptography Security goals, or CIA. These are our core security goals.
              \begin{itemize}
                  \item \textbf{Confidentiality:} only entitled entities may be able to read the data.
                  \item \textbf{Integrity:} tampering with a message will be detected.
                  \item \textbf{Authenticity:} the communicating parties can be identified/verified.
                \end{itemize}
    \end{itemize}
    

    %\begin{tabular}{|c|c|c|}
    %    \hline
    %    \textbf{URLLC Performance Metric} & \textbf{Minimal Value} & \textbf{Maximum Value} \\ \hline
    %    User plane latency (URLLC)        & -                      & 1ms                    \\ \hline
    %    Reliability (URLLC)               & 99.999\%               & -                      \\ \hline
    %\end{tabular}

    %\footnote{Minimum requirements related to technical performance
    %    for IMT-2020 radio interface(s), ITU, 2017 \cite{b1}}

\end{frame}



\section{Cryptography Security goals}
% 5
\begin{frame}
    \centering
    \includegraphics<1>[width=.5\textwidth, page=1]{pics/cia.png}
    \footnote{[Online; accessed 15-Jan-2024] https://www.xplore-dna.net/mod/page/view.php?id=6172}

    \begin{itemize}
        \item In addition, we might also have other goals.
              \begin{itemize}
                  \item \textbf{Non-repudiation:} a communication partner cannot deny a message originated from her.
                  \item \textbf{Forward secrecy:} compromise at this point in time does not lead to compromise of past connections.
                  \item \textbf{Availability:} systems and data are available to individuals when they need it under any circumstances, including power outages or natural disasters.
                \end{itemize}
    \end{itemize}
\end{frame}


\section{Private Keys}
% 6
\begin{frame}
    \centering
    \includegraphics<1>[width=.7\textwidth, page=1]{pics/private.png}
    \footnote{[Online; accessed 15-Jan-2024] https://en.wikipedia.org/wiki/Public-key_cryptography}

    \begin{itemize}
        \item We use encryption to provide confidentiality for the message m.
              \begin{itemize}
                  \item $k \ = \ Gen(1^n)$
                  \item $c \ = \ Enc_k(m)$
                  \item $m \ := \ Dec_k(c)$
                \end{itemize}
    \end{itemize}
    \begin{itemize}
        \item We use a message authentication code, e.g., HMAC-SHA2, to provide authenticity and integrity by generating a message authentication tag t for a message m:
              \begin{itemize}
                  \item $k \ = \ Gen(1^n)$
                  \item $t \ = \ Mac_k(m)$
                  \item $b \ := \ Vrfy_k(m, t)$
                \end{itemize}
    \end{itemize}
\end{frame}

\section{Public Keys}
% 7
\begin{frame}
    \centering
    \includegraphics<1>[width=.3\textwidth, page=1]{pics/public_key.png}
    \footnote{[Online; accessed 15-Jan-2024] https://en.wikipedia.org/wiki/Public-key_cryptography}

    \begin{itemize}
        \item Encryption is performed using the receiver's public key and decryption is performed by the receiver using his private key and vice versa.
              \begin{itemize}
                  \item $(pk, sk) \ = \ Gen(1^n)$
                  \item $c \ = \ Enc_{pk}(m)$
                  \item $m \ := \ Dec_{sk}(c)$
                \end{itemize}
    \end{itemize}
    \begin{itemize}
        \item Signatures are created using a private key and verified using the corresponding public key. They provide not only authenticity but also non-repudiation.
              \begin{itemize}
                  \item $(pk, sk) \ = \ Gen(1^n)$
                  \item $s \ = \ Sign_{sk}(m)$
                  \item $b \ := \ Vrfy_{pk}(m, s)$
                \end{itemize}
    \end{itemize}
\end{frame}

% 8
\section{Signatures and Certificates}
\begin{frame}
    \centering
    \includegraphics<1>[width=.6\textwidth, page=1]{pics/signature.png}
    \footnote{[Online; accessed 15-Jan-2024] https://cheapsslsecurity.com/blog/digital-signature-vs-digital-certificate-the-difference-explained/}
    \begin{itemize}
        \item Digital signatures are the cyberspace version of handwritten signatures.
        Both digital and handwritten signatures are sent with the original data to provide authentication.
    \end{itemize}
\end{frame}

% 9
\section{Transport Layer Security}
\begin{frame}
    \centering
    \includegraphics<1>[width=.7\textwidth, page=1]{pics/TLS.png}
    \footnote{[Online; accessed 15-Jan-2024] https://kinsta.com/de/blog/tls-1-3/}

    \begin{itemize}
        \item TLS stands for Transport Layer Security, is a protocol for securing layer 4 traffic and works on the session layer.
        \item The diagram above briefly summarizes the TLS 1.2 and TLS 1.3 handshakes.
        \item TLS 1.3 drops support for older, less secure cryptographic features. TLS 1.3 reduces TLS handshakes, 1-RTT instead of 2-RTT.
    \end{itemize}
\end{frame}

% 10
\section{Diffie-Hellman-Key-Exchange}
\begin{frame}
    \centering
    \includegraphics<1>[width=.8\textwidth, page=1]{pics/Diffie_Hellman_Key_Exchange.png}
    \footnote{[Online; accessed 15-Jan-2024] https://en.wikipedia.org/wiki/Diffie\%E2\%80\%93Hellman_key_exchange}

    \begin{itemize}
        \item Alice and Bob agree on two parameters $g$ and $p$ for the key export. This can be done by Alice simply choosing them and sending them to Bob.
        \item Both then generate a random number. This number is known only to them and will not be sent anywhere. Using this random number, Alice calculates $A \ = \ g^a \ mod \ p$ and Bob calculates $B \ = \ g^b \ mod \ p$. A and B are then sent to their respective other partners.
        \item They can now both generate the shared key K by computing $K \ = \ B^a \ mod \ p$ and $K \ = \ A^b \ mod \ p$.
    \end{itemize}
\end{frame}

% 11
\section{Certificate Authority}
\begin{frame}
    \centering
    \includegraphics<1>[width=.6\textwidth, page=1]{pics/PKI_Chain_of_Trust.png}
    \footnote{[Online; accessed 15-Jan-2024] https://www.semanticscholar.org/paper/PKI-trust-relationships\%3A-from-a-hybrid-architecture-Satiz\%C3\%A1bal-P\%C3\%A1ez/4e1d85f070b4458b6769d5f8fb976003a1b4fe45}
    \begin{itemize}
        \item A client requests a service from a server.
        \item Since it is a TLS-protected session, the client asks the server to prove its identity 
        with a certificate. It is signed by a $CA_n$. The client has never heard of $CA_n$ and obtains 
        a certificate from $CA_n$. It is signed by $CA_{n-1}$. Since the client only knows the root CA, 
        it must obtain all certificates until it retrieves one signed by the root CA.
    \end{itemize}
\end{frame}

% 12
\section{Introduction of WoT}
\begin{frame}
    \centering
    \includegraphics<1>[width=.8\textwidth, page=1]{pics/decentralized_trust_model.png}
    \footnote{[Online; accessed 15-Jan-2024] https://de.wikipedia.org/wiki/Web_of_Trust}
    \begin{itemize}
        \item Web of Trust (WoT) is a concept in cryptography that can be used to authenticate the identity of the
        holder of a public key, as applied in PGP, GnuPG, or other OpenPGP-compatible systems.
        \item Trust networks use the concept of decentralization.
    \end{itemize}
\end{frame}

% 13
\section{Direct Trust}
\begin{frame}
    \centering
    \includegraphics<1>[width=.6\textwidth, page=1]{pics/Acc&Pre2}
    \footnote{[Online; accessed 24-Jan-2023] https://sixsigmadsi.com/precision-and-accuracy/}
    \begin{itemize}
        \item Accuracy: the timestamp differences between measurement and occurrence
        \item Precision: the fluctuations between measurements
    \end{itemize}
\end{frame}

% 14
\section{Introduction of WoT}
\begin{frame}
    \centering
    \includegraphics<1>[width=.6\textwidth, page=1]{pics/Acc&Pre2}
    \footnote{[Online; accessed 24-Jan-2023] https://sixsigmadsi.com/precision-and-accuracy/}
    \begin{itemize}
        \item Accuracy: the timestamp differences between measurement and occurrence
        \item Precision: the fluctuations between measurements
    \end{itemize}
\end{frame}

% 15
\section{Introduction of WoT}
\begin{frame}
    \centering
    \includegraphics<1>[width=.6\textwidth, page=1]{pics/Acc&Pre2}
    \footnote{[Online; accessed 24-Jan-2023] https://sixsigmadsi.com/precision-and-accuracy/}
    \begin{itemize}
        \item Accuracy: the timestamp differences between measurement and occurrence
        \item Precision: the fluctuations between measurements
    \end{itemize}
\end{frame}

% 16
\section{Trust in Social Networks}
\begin{frame}
    \centering
    \includegraphics<1>[width=.6\textwidth, page=1]{pics/Acc&Pre2}
    \footnote{[Online; accessed 24-Jan-2023] https://sixsigmadsi.com/precision-and-accuracy/}
    \begin{itemize}
        \item Accuracy: the timestamp differences between measurement and occurrence
        \item Precision: the fluctuations between measurements
    \end{itemize}
\end{frame}

% 17
\section{Experimental Results and Analyses}
\begin{frame}
    \centering
    \includegraphics<1>[width=.4\textwidth, page=1]{pics/Optimization2}
    \includegraphics<1>[width=.55\textwidth, page=1]{pics/Optimization1}
    \footnote{S. Gallenmüller, F. Wiedner, J. Naab, and G. Carle, “Ducked Tails:
        Trimming the Tail Latency of(f) Packet Processing Systems,” 17th
        International Conference on Network and Service Management
        (CNSM), 2021, p.537, 2021 \cite{b5}}
    \begin{itemize}
        \item Disable virtual cores
        \item Enable core isolation
        \item Reduce the number of CPU OS interrupts
        \item Disable energy saving mechanism
    \end{itemize}
\end{frame}

% 18
\section{Conclusion}
\begin{frame}
    \centering
    \includegraphics<1>[width=.4\textwidth, page=1]{pics/Optimization3}
    \includegraphics<1>[width=.4\textwidth, page=1]{pics/Optimization4}
    \footnote{A. M. Joy, “Performance Comparison between Linux Containers
        and Virtual Machines,” International Conference on Advances in
        Computer Engineering and Applications, 2015, p. 342, 2015. \cite{b4}}
    \begin{itemize}
        \item Containers:
              \begin{itemize}
                  \item over 5 times requests more than VMs
                  \item great scalability, about 22 times of VMs
                        (high portability)
              \end{itemize}
    \end{itemize}
\end{frame}

