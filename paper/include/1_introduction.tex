\section{Introduction}
Today, global digitization continues to grow at an astonishing rateand the related investments keep 
growing.  The concepts like Web 3, Meta-Universe, Bit Coin, Block Chain etc. have become hot topics of 
the day - it is estimated that the digital economy will account for more than 60\% 
of global GDP in recent years\cite{b15}. However, while the digital economy is booming, the network 
ecosystem behind it is undergoing fundamental changes. The original globalized cyber 
architecture is gradually disintegrating under the combined effect of weaponization of 
cyberspace and conflicting interests of various countries, evolving into a regionalized 
or even nationalized architecture. The world will see unprecedented challenges to the resilience 
of global networks and systems\cite{b5}.
\\
At the same time, in addition to the traditional network communication security, 
major enterprises and end-users have generated new security needs, such as lower costs, 
more privacy, more transparent regulation etc.
\\
With the development, information security gradually produced two kinds of security 
system, centralized and decentralized security system. Their corresponding 
representative technologies, namely PKI - Public Key Infrastructure and WoT - Web of Trust.
\\
PKI is a system of centralized processes, technologies and policies that enable users 
to encrypt and sign data\cite{b5}. A third-party authority can issue digital certificates that 
authenticate a user, device or service. These certificates can create secure connections 
for public Web pages and private systems, where private systems include private virtual 
private networks (VPNs), internal Wi-Fi, Wiki pages, and other MFA-enabled services\cite{b6}.
\\
On the other hand, WoT is a concept in cryptography used to authenticate the identity 
of the holder of a public key, as applied in PGP\cite{b7}, GnuPG\cite{b8}, or other OpenPGP-compatible 
systems\cite{b9}. WoT is based on the concept of decentralization, which is different from public key 
infrastructures relying on third-party digital certificate certification authorities. 
In a computer network, there can be many independent WoTs at the same time, and any user 
can be a part of these networks, or a link between different networks\cite{b7}.
\\
In today's time, Public Key Infrastructure (PKI) has become one of the most mature 
encryption technologies\cite{b5}.
\\
With the development of the times, in addition to the encryption security in the 
process of network communication, users have generated new security needs, for example, 
users are more and more concerned about the privacy protection of their own identity information.
\\
In other words, PKI, a centralized encryption method that relies on a third-party 
authority, is contrary to the above needs. In addition, the PKI-based security system still has 
great potential security risks\cite{b6}.
\\
Therefore, people need a new decentralized secure communication method, or a decentralized 
encryption method. Currently, the increasingly mature WoT technology meets this demand of 
the new age users\cite{b14}.
\\
However, human beings are suspicious and insecure. For security reasons, people tend 
to prefer old, mature technologies, such as the most traditional PKI system. Getting 
most people to trust a new, unfamiliar technology takes a lot of time. From understanding 
it, to applying it, to validating it in various real-world scenarios, it is a long and 
tedious process. The purpose of this paper is to help readers have an initial 
understanding and impression of WoT, the emerging cryptographic technology of decentralization. 
It also analyzes the future development trend of network security by comparing the two 
security systems based on WoT and PKI.