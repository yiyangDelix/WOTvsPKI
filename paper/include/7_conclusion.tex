\section{Conclusion}
In this paper, we propose a user context-based trust level security system in social 
network services, which is introduced into network services through the process of 
generating trust among people in sociology and psychology, and is used to extend the 
trust network among digital users\cite{b22}. The trust between users is categorized into three 
levels: complete trust, suspicion and complete distrust\cite{b10, b14}. Again, depending on the level 
of trust, user pairs are connected to each other to form a larger social trust network\cite{b20}. 
Finally, the reasonableness and effectiveness of the approach is verified by searching 
the literature and simulation experiments through the comparison of two security systems 
based on WoT and PKI\cite{b15, b16}.
\\
The purpose of this article is to help readers have a preliminary understanding and 
impression of WoT, an emerging cryptographic technology for decentralization\cite{b19}. This 
paper shows that the use of traditional PKI security system in today's digital platforms 
is not perfect. PKI poses a great potential risk to the user's privacy and security\cite{b21}. 
A decentralized security system based on Web of Trust can maximize the protection of 
communication between two users. This is more suitable for today's users and is more 
in line with the sociological and psychological process of building trust between people\cite{b24}.